%!TEX TS-program = xelatex 
%!TEX TS-options = -output-driver="xdvipdfmx -q -E"
%!TEX encoding = UTF-8 Unicode
%
%  enactedcolor
%
%  Created by Mark Eli Kalderon on 2008-07-01.
%

\documentclass[12pt]{article} 

% Definitions
\newcommand\mykeywords{color, constancy, Noë, enacted perception, perception} 
\newcommand\myauthor{Mark Eli Kalderon} 
\newcommand\mytitle{Color and the Problem of Perceptual Presence}

% Packages
\usepackage{geometry} \geometry{a4paper} 
\usepackage{url}
\usepackage{pdfsync} 
\usepackage{txfonts}
\usepackage{graphicx}
\usepackage{color}
\definecolor{gray}{rgb}{0.459,0.438,0.471}
% \usepackage{setspace}
% \doublespace % Uncomment for doublespacing if necessary
% \usepackage{epigraph} % optional

% XeTeX
\usepackage[cm-default]{fontspec}
\usepackage{xltxtra,xunicode}
\defaultfontfeatures{Scale=MatchLowercase,Mapping=tex-text}
\setmainfont{Hoefler Text}
\setsansfont{Gill Sans}
\setmonofont{Inconsolata}

% Section Formatting
\usepackage[]{titlesec}
\titleformat{\section}[hang]{\fontsize{14}{14}\scshape}{\S{\thesection}}{.5em}{}{}
\titleformat{\subsection}[hang]{\fontsize{12}{12}\scshape}{\S{\thesubsection}}{.5em}{}{}
\titleformat{\subsubsection}[hang]{\fontsize{12}{12}\scshape}{\S{\thesubsubsection}}{.5em}{}{}

% Headers and Footers
\usepackage{fancyhdr}
\pagestyle{fancy}
\pagenumbering{arabic}
\lhead{\thepage}
\chead{}
\rhead{\itshape{\nouppercase{\leftmark}}}

% Bibliography
\usepackage[round]{natbib} 

% Title Information
\title{\mytitle}% Thanks optional \thanks{Thanks to Keith Allen, MGF Martin, and Charles Travis.}
\author{\myauthor} 
% \date{} % Leave blank for no date, comment out for most recent date

% PDF Stuff
\usepackage[plainpages=false, pdfpagelabels, bookmarksnumbered, backref, pdftitle={\mytitle}, pagebackref, pdfauthor={\myauthor}, pdfkeywords={\mykeywords}, xetex, dvipdfmx, colorlinks=true, citecolor=gray, linkcolor=gray, urlcolor=gray]{hyperref} 

%%% BEGIN DOCUMENT
\begin{document}

% Title Page
\maketitle
% \begin{abstract} % optional
% \end{abstract} 
\vskip 2em \hrule height 0.4pt \vskip 2em
% \epigraph{text of epigraph}{\textsc{author of epigraph}} % optional; make sure to uncomment \usepackage{epigraph}

% Layout Settings
\setlength{\parindent}{1em}

% Main Content

\section{Perceptual Constancy}\label{sec:perceptual_constancy} % (fold)

Very often, objects in the scene before us are somehow perceived to be constant or uniform or unchanging in color, shape, size, or position, even while their appearance with respect to these features somehow changes. This is a familiar and pervasive fact about perception, even if it is notoriously difficult to describe accurately let alone adequately account for. These difficulties are not unrelated---how we are inclined to \emph{describe} the phenomenology of perceptual constancy will affect how we are inclined to \emph{account} for it.

\citet{Noe:2004fk} believes that the facts of color constancy, that the color of a thing appears to persist unaltered though it presents distinct appearances, speaks in favor of a particular view about color experience and a particular view about the metaphysics of color. Though we experience the persistent unaltered color, we strictly speaking see only the variable \emph{look} or \emph{appearance}. Moreover, colors are nothing over and above the patterns of variable looks or appearances.

Though there is much to admire in Noë's account of perception---particularly noteworthy is his resistance to any conception of experience as a form of inner representation, nevertheless, his views about color experience and the metaphysics of color are inadequate. Specifically, they fail to account for the very phenomena that motivates them. The root of this difficulty is that Noë has misdescribed color constancy at the outset. A better description of the phenomenology motivates a naïve realism about color experience and the metaphysics of color, a naïve realism that retains many of the virtues that Noë claims for his own account \citep[see][for similar suggestions]{Allen:2008kx,Campbell:2008nx}.

% section perceptual_constancy (end)

\section{The Problem of Perceptual Presence}\label{sec:the_problem_of_perceptual_presence} % (fold)

According to Noë, we experience more than we ``strictly speaking'' see. 

When I look at the tomato ripening on my window sill, I experience a voluminous whole even though I strictly speaking see only the frontside of the tomato \citep[76]{Noe:2004fk}. When I look at Ricca's cat as he passes behind the leg of the kitchen table, I experience the whole cat even though, I strictly speaking see only those parts of the cat not obscured by the table leg \citep[60]{Noe:2004fk}. The same goes for constancy phenomena. When I look at the plate on the table, I experience the plate as circular even though, from my current vantage point, I strictly speaking see only its elliptical look \citep[78--79]{Noe:2004fk}. And when I look up at the unevenly illuminated wall, I experience the wall as uniformly white even though I strictly speaking see only gradations of gray \citep[127]{Noe:2004fk}. 

What is it to experience voluminous wholes, real shapes, and uniform colors? It is to have a practical understanding of how the look or appearance of things varies with the conditions under which they are perceived:
	\begin{quote}
		The experience of shape depends on our implicit grasp of the way perspectival shape varies as we move in respect to an object. We don’t have names for every aspect we encounter, but we have a grip on the way aspects vary. This grip is, in effect, our grasp of what it is for something to be presented as cubical, or spherical. It is much harder to make out what our grasp on the form of a shoulder, or a human jaw, or a hip, consists in. But there is no reason, in principle, why it cannot consist in something very much like this \ldots\ Similar kinds of considerations \ldots\ go for color: Our grasp of color depends on our implicit mastery of the way appearances change as color critical conditions change. \citep[198--199]{Noe:2004fk}
	\end{quote}
This analysis is developed, refined, and extended (for example, to account for Molyneux's Question) throughout \emph{Action in Perception}. I will have more to say about it presently. But for now, I want to consider what could motivate this analysis in the first place. Why think that we experience more than we strictly speaking see?

Sometimes Noë seems motivated by a dogmatic adherence to an alleged phenomenological datum---that it is just \emph{obvious} that we do not strictly speaking see the tomato that we experience. But if it is obvious, then why do so many theorists of perception disagree? It cannot be that they are \emph{all} misled by the picture of experience as an inner representation \citep[chapter 2]{Noe:2004fk}, for not all who disagree accept that picture---naïve realists and disjunctivists maintain that veridical perception is relational and hence nonrepresentational and yet claim that among the objects that sight affords awareness of are voluminous wholes such as garden variety tomatoes. Nor is it scarcely plausible that a plurality of alien mentalities pervade among theorists of perception. It is not as if we may conclude, as \citet[69]{Sartwell:1995ve} wryly puts it, that ``some people \ldots\ are appeared to pigly, [while] other[s] have pigs thrust upon them''. 

I believe that if we attend closely to Noë's description of perceptual phenomena we can discern a line of thought that can be reconstructed as an argument. When we do, the argument turns out to be a variant of an ancient argument, \emph{the argument from conflicting appearances}. In this way, Noë is following the example of \citet{Berkeley:1734fk} and sense data theorists such as \citet{Russell:1912uq} and \citet{Price:1932fk} in understanding constancy phenomena as giving rise to conflicting appearances or, at least, a conflict within appearance.

% section the_problem_of_perceptual_presence (end)

\section{Conflicting Appearances and the Conflict Within Appearance}\label{sec:conflicting_appearances} % (fold)

Some evidence for this suggestion is provided by Noë himself:
	\begin{quote}
		When you look at the wall, you see its uniform color \emph{in} its evident variation in color across its surface. When you look at a circular plate, held up at an angle, you experience its circularity \emph{in} its merely elliptical shape. When you look at a tomato, you experience it as full-bodied and three dimensional even though you don't see its sides or back; you experience its three-dimensionality in its visible parts. Part of what makes the study of perception so difficult is the necessity of acknowledging not only this dual aspect in perceptual content, \emph{but the prima facie conflict in perceptual content}. [my emphasis]
	\end{quote}
Indeed, it is plausible that Noë thinks that the necessity in acknowledging the dual aspect in perceptual content is due to the way it resolves the prima facie conflict in perceptual content.

Wherein is the alleged prima facie conflict in perceptual content?

Since our topic is perceptual constancy, and color constancy in particular, let's focus on these cases.

Our experience of a colored object can vary as the circumstances of perception varies. Our experience of a colored object can vary as the object's position relative to the illuminant varies, as when that object is now more, now less directly illuminated \citep[125]{Noe:2004fk}. Our experience of a colored object varies as the nature of the illuminant changes, as when we view that object now in artificial illumination, now in natural daylight \citep[125]{Noe:2004fk}. Our experience of a colored object can vary as we change our position to the object while it's position to the illuminant remains constant, as when we experience shifting shadows and specular highlights as we move around a stable object \citep[125]{Noe:2004fk}. Our experience of a colored object can vary as the color of the surrounding scene varies due to color contrast effects \citep[125--127]{Noe:2004fk}---effects systematically explored in the paintings of \citet[]{Albers:1963gf}. It is important to emphasize, as \citet[62-63]{Cohen:2008hc} observes, that the variable appearance of a uniformly colored object can be presented in a single experience of that object as well as in distinct experiences of that object in distinct circumstances of perception, as when the uniformly colored object is viewed under uneven illumination. Color constancy can be \emph{simultaneous} as well as \emph{successive}.

Roughly speaking, the prima facie conflict in perceptual content is the appearance of an uniform color despite the variable color appearances:
	\begin{quote}
		Despite these changes in apparent color, and despite specular colors, perceivers are usually able to recognize, say, that the car is a uniform and unchanging red. \ldots\ We see the uniformity despite, or behind or beneath (as it were), the variable appearance. We do not confuse changes in the apparent color as color critical conditions change with changes in the underlying actual color. \ldots\ We experience color as that which is, in a wide range of cases, \emph{invariant} amid the apparent variation. \citep[127]{Noe:2004fk}
	\end{quote}

It is important to emphasize that this conflict \emph{could only be prima facie}. Consider cases of simultaneous color constancy, as when an uniformly colored surface is unevenly illuminated. If the conflict were genuine, then our experience of the uniform color despite its variable appearance would be like an experience of an impossible scene, such as Penrose's \citeyearpar{Penrose:1958kx} impossible triangle (see \autoref{fig:triangle}), or a scene depicted by an Escher drawing.
	\begin{figure}[htbp]
		\centering
			\includegraphics[scale=1]{triangle.jpg}
		\caption{Impossible Triangle}
		\label{fig:triangle}
	\end{figure}
However, in cases of simultaneous color constancy, our experience of the invariant color amid the variable appearances is not incoherent in this way, nor does it have the requisite tension that results from such incoherence. Nor does successive color constancy give rise to any visual puzzlement. And that means that the apparent conflict within or between experience could only be apparent. The conflict arises solely in how we might be tempted to describe the phenomenology of stability and flux. 

What, then, more precisely, is the prima facie conflict in perceptual content? 

A closer look at some of Noë's examples is revealing:
	\begin{quote}
		For example, suppose you enter a room and see that the wall is a uniform shade of white. You also see that the wall is brighter here, where it falls in direct sunlight, than it is there, where it falls in shadow. Differences in brightness, however, mean differences in color. You see the uniformity of color despite the evident nonuniformity of different parts of the wall's surface \citep[127]{Noe:2004fk}
	\end{quote}
and again:
	\begin{quote}
		Crucially, we can experience the wall as uniform in color \emph{and} as differently colored across its surface. Just as we can see that the plate looks circular \emph{and} elliptical, so we can see the color is uniform \emph{and} variable. \citep[129]{Noe:2004fk}
	\end{quote}

Specular highlights are another case:
	\begin{quote}
		For example, the specular highlights on the surface of a clean, new automobile vary as viewing geometry changes \ldots\ As you move in relation to the car, or as it moves in relation to you, the apparent color of the car's surface may visibly change. \citep[125]{Noe:2004fk}
	\end{quote}
And yet the color of the car appears uniform and unaltered with the change in viewing geometry.

The prima facie conflict, then, is this: The wall appears uniformly white and yet parts of the wall appear gray. The car appears uniformly red and yet parts of the car appear white. But nothing can have a visible surface that is uniformly white and partly gray and nothing can have a visible surface that is uniformly red and partly white. So we have a prima facie conflict in perceptual content.

This prima facie conflict in perceptual content grounds the necessity in acknowledging the dual aspect in perceptual content. The attribution of the dual aspect is meant to resolve this prima facie conflict. On the one hand, what we strictly speaking see are \emph{apparent colors}. The unevenly illuminated white wall varies in apparent color. (This is evidently a technical term since it departs from ordinary usage. Ordinarily, the apparent color of an object is the color it appears to have. On this, ordinary, understanding of the phrase, the apparent color of the wall is white, even where it is shadowed---so long as the difference in illumination is within the bounds of human color constancy). Apparent colors, for Noë, are objective if relational properties of things. A white wall in shadow has the relational property, being gray in shadow. On the other hand, while we strictly speaking see apparent colors, we do not strictly speaking see colors, though we experience them thanks to our practical understanding of how apparent colors change in different circumstances of perception, or ``color critical'' conditions. Colors themselves are not objective if relational properties of things, but are rather patterns of these relational looks or appearances, or ``color aspect profiles''. While nothing can have a visible surface that is uniformly white and partly gray, there is nothing inconsistent with something having a visible surface that is uniformly white a portion of which has the relational property apparent gray. Indeed part of what it is for the wall to be white is for it to manifest apparent grayness when in shadow.

% section conflicting_appearances (end)

\section{The Phenomenology of Stability and Flux}\label{sec:stability_and_flux} % (fold)

We will examine the merits of this proposal in sequel. For now, we will consider only whether Noë has adequately described the prima facie conflict that motivates it. If no conflict is generated from an adequate description of the phenomenology of stability and flux, then acknowledging the dual aspect in perceptual content is unnecessary.

\subsection{Shadows} % (fold)
\label{sub:shadows}

Let's begin with the shadowed white wall. The part of the white wall cast in shadow appears differently from the part directly illuminated, yet all of the parts appear to be uniformly white. How are we to characterize the difference? Though the parts are uniformly colored they differ in appearance. Moreover this, Noë insists, is a chromatic difference. The apparent color of the shadowed wall is some shade of gray. Notice this is what is required to generate the prima facie conflict in this case---that no visible surface can be uniformly white and partly gray. Noë offers two reasons for thinking that the shadowed white wall appears gray. Neither are convincing.

First, Noë appeals to the difference in brightness between the shadowed parts and the parts that are directly illuminated:
	\begin{quote}
		You also see that the wall is brighter here, where it falls in direct sunlight, than it is there, where it falls in shadow. Differences in brightness, however, mean differences in color. You see the uniformity of color despite the evident non-uniformity of different parts of the wall's surface. \citep[127]{Noe:2004fk}
	\end{quote}
The reasoning, here, can seem to turn on an equivocation. It is true that the wall is brighter where it is directly illuminated where ``brightness'', here, means the intensity of the light reflected by the white wall in direct sunlight. However, Noë needs to understand ``brightness'' as a dimension of color similarity and difference, if he is to argue from a difference in brightness to a difference in color. Unfortunately these distinct senses of brightness can come apart. Consider viewing a page of black print on white paper, first indoors under artificial illumination, then outside in natural daylight. The intensity of the light reflected by the white area of the page indoors is approximately the same as the intensity of the light reflected by the black print in sunlight \citep[199]{Peter-K:1996th}. Despite being equally bright, in the sense that the reflected light is equally intense, the apparent colors differ in chromatic brightness---the surrounding white when viewed indoors seems brighter than the black print when viewed in daylight.

Perhaps, Noë thinks that it is evident that wall in direct sunlight is brighter not only in intensity of reflected light, but also chromatically brighter. Is this really evident? He offers the following consideration:
	\begin{quote}
		Consider that although we can perceive a wall that is illuminated unevenly as uniform in color, it is also the case that when a wall is in this way illuminated unevenly, it is also visibly different with respect to color across its surface. For example to match the color of different parts of the wall, you would need different color chips. \citep[128]{Noe:2004fk}
	\end{quote}
Since we can match a gray chip to how the shadowed wall appears, the apparent color of the shadowed wall is gray. 

But what kind of a matching task are we meant to be engaged in? Suppose you are looking at the wall in question and you interpose the chip so that their apparent color may be compared. You might say that the wall looks gray, but it would wrong to conclude that the shadowed wall and the chip share the same appearance for their would remain a visual difference. Austin makes the point this way:
	\begin{quote}
		Again, it is simply not true to say that seeing a bright green after-image against a white wall is exactly like seeing a bright green patch actually on the wall; or that seeing a white wall through blue spectacles is exactly like seeing a blue wall; or that seeing pink rats in D.T.s is exactly like really seeing pink rats; or (once again) that seeing a stick refracted in water is exactly like seeing a bent stick. In all these cases we may \emph{say} the same things (`It looks blue', `It looks bent', \&c.), but this is no reason at all for denying the obvious fact that the `experiences' are different. \citep[49]{Austin:1962lr}
	\end{quote}

A straight stick submerged in water and a bent stick may each look bent. But the straight stick in water looks different from a bent stick in at least one important respect---the appearance of an intervening medium through which the stick is viewed. There is a parallel in the present case. The part of the wall cast in shadow and the chip may each look gray. But the shadowed wall and the chip look different in at least one important respect---the appearance of the interviewing medium. Whereas the intervening medium in the case of the stick is water, the intervening medium in the case of the wall is, of course, the shadow. Not only do we see the colors of things, but we also see the way they are illuminated \citep[see][]{Hilbert:2007qy}.

Noë evidently shares the intuition with \citet[88]{Chalmers:2006kx}, that a shadowed white thing looks the same as an unshadowed gray thing. While we may grant that a shadowed white thing looks like an unshadowed gray thing, it is controversial that a shadowed white thing looks exactly like an unshadowed gray thing. (This is an application of Austin's point which Noë, \citeyear[80]{Noe:2004fk}, notes, but makes nothing of, nor notices its present application.) Indeed there is a demonstrable difference. Consider Hering's \citeyearpar[8]{Hering:1920ty} ringed shadow experiment. When a shadow is cast on a surface you see the shadow as ``an incidental darkness that lies on the [surface]''. But the darkness is not seen as a property of the surface. Suppose we draw a black line around the shadow that completely obscures its penumbra. The darkness no longer appears to lie on the surface, but rather appears to be a property of the surface. Then, and only then, does the shadowed white surface appear to be gray. A shadowed white thing, with an unobscured penumbra, does not appear to be gray, though it may, in some respects, look like an unshadowed gray thing. Indeed, Hilbert offers the following physical explanation of this similarity:
	\begin{quote}
		A gray surface is one that is one whose reflectance is non-selective, i.e.\ fairly constant across the visible spectrum, and intermediate, i.e.\ lower than white but higher than black. Shading is a non-selective, partial decrease in the amount of light falling on a surface. If the visual system were to represent surface grays in terms of proportions of the best white and to represent illuminants in terms of proportions of the average illumination for the scene then there would be a formal analogy between the representation of gray surfaces and shaded surfaces. In addition, there is a similarity in the content of the representations in that both are concerned with the object’s relation to the light in its environment. In other words, a gray surface is one that reflects significantly less than all the available light, while a shaded surface is one that is illuminated by significantly less than all the available light. This similarity in content may be enough to account for the similarity in phenomenology. \citep[xx]{Hilbert:2007qy}
	\end{quote}
If we see, not only the color of an object, but also the way it is illuminated, then this explanation of the similarity is consistent with the demonstrable difference.

The problem, then, is this: The case of the shadowed wall, properly described, gives rise to no prima facie conflict in perceptual content. The prima facie conflict was meant to be that the wall appears uniformly white and partly gray. But is simply not true that the part of the wall cast in shadow appears gray. It may look, in some respects, like an unshadowed gray thing, but a visual difference remains. And with no prima facie conflict in perceptual content, acknowledging the dual aspect in perceptual content is unnecessary.

% subsection shadows (end)

\subsection{Specular Highlights} % (fold)
\label{sub:specular_highlights}

What about cases of perceptual constancy involving specular highlights? In \emph{The Problems of Philosophy}, Russell writes:
	\begin{quote}
		Although I believe that the table is ``really'' of the same colour all over, the parts that reflect the light look much brighter than the other parts, and some parts look white because of the reflected light. \citep[2]{Russell:1912uq}
	\end{quote}
Specular highlights may look brighter than the rest of the surface in the sense of the intensity of the light reflected. Specular highlights may be white set against an expanse of red. But does the part of the surface corresponding to the specular highlight itself look brighter than the rest of the surface? Is it even seen? Or is that part of the surface rather \emph{occluded} by a white specular highlight. The whiteness would then be the color of the reflected light, not, as Russell suggests, the surface. There is an analogous case: The surface of a mirror seen under intense illumination may be obscured by the glare (as when a distant compatriot signals with a mirror in the desert). If the white of the specular highlight occludes the red of the surface, then that part of the surface corresponding to the specular highlight does not appear white---it doesn't even appear! If I were interested in some detail of that part of the surface corresponding to the specular highlight, I would need to change the viewing geometry (either by altering my position to the object or by altering the object's position to the illuminant) to get that detail into view.

Noë is not misled the way Russell is. He writes:
	\begin{quote}
		Specular highlights are frequently the color of the incident light itself, reflecting white in the sunshine, and also the colors of, for example, street lights. \citep[125]{Noe:2004fk}
	\end{quote}
However, he continues:
	\begin{quote}
		As you move in relation to the car, or as it moves in relation to you, \emph{the apparent color of the car's surface may visibly change}. [my emphasis] \citep[125]{Noe:2004fk}
	\end{quote}
Recall, as ordinarily understood, the apparent color of a thing is the color that if appears to have. If the car is red and in plain view to a normal perceiver, then the apparent color of the car is red despite its variable appearance due to shadows, reflections, and specular highlights. So, as ordinarily understood, it is simply not true that the apparent color of the car's surface may visibly change as you move in relation to the car. To claim otherwise would be to claim that the car appears to change color as you change your relative position. This, in turn, would involve Russell's mistake of counting the color of the specular highlight as the color of the car. 

Perhaps Noë has in mind the technical sense of ``apparent color'', where apparent colors are objective if relational looks or appearances. This would fit well the phenomena of specular highlights. As \citet[141]{Johnston:1992ck} observes that specular highlights ``reveal their relational nature'':
	\begin{quote}
		They change as the observer changes position relative to the light source. They darken markedly as the light source darkens. With sufficiently dim light they disappear while the ordinary color remain.
	\end{quote}

	These two interpretations of ``apparent color'' form the basis of a dilemma:
	\begin{itemize}
		\item If ``the apparent color'' of the car is ordinarily understood, the the color of the car appears to change as you change your relative position. This would generate the prima facie conflict, but it would also misdescribe the phenomenology of color constancy. 
		\item If ``the apparent color'' of the car is understood in the technical sense, then it does not misdescribe the phenomenology of color constancy in the same way, but no prima facie conflict is generated. 
	\end{itemize}
And if no prima facie conflict is generated, with what right does Noë invoke this aspect of the dual content of perceptual content? It is question-begging to appeal to objective if relational looks or appearances at this point in the discussion.

Perhaps the prima facie conflict in perceptual content is generated in another way. There is a sense in which the case of specular highlights is analogous to the case of seeing voluminous wholes. If the white is the color of the reflected light, then the color of the surface corresponding to the specular highlight is occluded. We don't see the color of that part of surface any more than we see the backside of a tomato. 

When I look at the tomato ripening on my window sill, of all the parts of the tomato that I see, I see only the frontside of the tomato. Its backside and interior are hidden from me, at least given my present vantage point and the present, unsliced, state of the tomato. Not only do I see the frontside of the tomato, I experience the tomato itself as a voluminous whole. As noted earlier, Noë maintains that while I strictly see only the frontside of the tomato I experience the tomato as a voluminous whole thanks to my practical understanding of how the look or appearance of the tomato will vary as the conditions under which it is perceived vary. That is the dual aspect in the content of my perception of the tomato. What is the prima facie conflict that grounds the necessity of acknowledging the dual aspect of this perceptual content?

Of all the parts of the tomato that I see, I see only the frontside of the tomato. This is an instance of a familiar fact about the visibility of opaque objects. I also see the voluminous whole. \emph{So far there is no conflict}. In order to generate the prima facie conflict in perceptual content, we would need to assume, in addition, that we can only see a whole by seeing all of its parts. This is Chisholm's \citeyearpar[154--156]{Chisholm:1957dq} diagnosis of why some, such as Moore, have been tempted to claim that we strictly speaking see only the surfaces of opaque objects:
\begin{quote}
	Whenever we talk roughly of seeing any object, it is true that, in another and stricter sense of the word \emph{see}, we only see \emph{a part of} it. \citep[34]{Moore:1953nx}
\end{quote}
According to Chisholm, then, the conflict is only generated if we assume that the visible is \emph{dissective} in something like Goodman's \citeyearpar[48-49]{Goodman:1951ww} sense of the term---that we can only see a whole by seeing all of its parts. 

Notoriously, \citet{Clarke:1965yg} denies that theorists like Moore have been misled by the hidden assumption that the visible is dissective---in certain contexts we \emph{do} only see the frontside of the tomato. However, whatever the merits of Clarke's case, it is of no help at present. The occasion-sensitivity of perception is a way of \emph{dissolving} the prima facie conflict in perceptual content and not a way of generating it.

Recently, \citet{Martin:2008kl} has argued that Chisholm's diagnosis is applicable to Noë's case as well. Noë never explicitly articulates the dissectivity of the visible (though he comes close at places). Moreover, \citet[698-699]{Noe:2008oq} explicitly \emph{denies} that the visible is dissective in his response to Martin. However, as Quine continually reminds us, belief is one thing, and commitment quite another. There is simply no way to generate the alleged prima facie conflict in perceptual content without making some such assumption.


Consider again the way the color of the specular highlight obscures the color of the corresponding part of the surface. I see the color of the the specular highlight. I see the uniform color of the surface. \emph{So far there is no conflict.} In order to generate the prima facie conflict in perceptual content, we would need to assume, in addition, that color experience is dissective---that we can only see the color of a surface by seeing the color of all its parts. But if the dissectivity of color experience, along with the evident facts of perception (that I see the color of the specular highlight and I see the color of the surface), is what generates the prima facie conflict in perceptual content, then we should reject the assumption that color experience is dissective. Moreover, rejecting that assumption does not require that we attribute to color experience a dual content. Rather---to anticipate a theme to be developed in sequel---as \citet{Hilbert:1987jq} has emphasized, rejecting the assumption that color experience is dissective requires only that we acknowledge that color perception is \emph{partial}.\\

% subsection specular_highlights (end)

\noindent Noë claims that the necessity in acknowledging the dual aspect in perceptual content is grounded in the prima facie conflict in perceptual content. However, at least with respect to color constancy, there is no prima facie conflict. To that extent, at least, the attribution of a dual aspect in content to color experience is unmotivated.

% section stability_and_flux (end)

\section{Illumination Insensitivity and the Vestigial Remnants of Inner Representation} % (fold)
\label{sec:vestigal_remnants_of_inner_representations}

One of the many salutary aspects of \emph{Action in Perception} is Noë's trenchant opposition to the inner representation model of experience. However, philosophical pictures have a life of their own and are not so easily quelled. Indeed, many of the problems of the previous section are due to the lingering effects of the inner representation model.

Experience, as conceived by Noë, is illumination insensitive in the following sense: Noë is implicitly committed to our not strictly speaking seeing illumina\-tion-dependent properties, which is not, of course, to say that they do not figure in our experience. Moreover, this illumination insensitivity is arguably a vestigial remnant of inner representation. For on Noë's account, experience is, if not an inner picture, then, in an important respect, picture-like---illumination-dependent properties figure in our experience the way that they are depicted by pictures. 

To bring this out, let's briefly reconsider three cases of color constancy.

Being shadowed is an illumination-dependent property of things. For a thing to be shadowed is for that thing to have a shadow cast upon it. Shadows themselves are species of privation---they are like holes in the light cast upon things \citep[see][]{Sorensen:2008kx}. As we have seen, the distinctive appearance of the white wall cast in shadow is not the apparent color of the wall. At least as ordinarily understood, the apparent color of the wall, the color that it appears to have, is white. Nor is it the apparent color of the shadow. Arguably at least, the color that shadows appear to have is black \citep{Sorensen:2008kx}, even though this appearance is obscured somewhat by light pollution. The distinctive appearance of the white wall cast in shadow is the way that white appears when cast in shadow. Compare Austin's \citeyearpar{Austin:1962lr} own example of perceptual constancy---a straight stick submerged in water does not appear bent; it appears the way that a straight stick appears when submerged in water. Understanding the appearance of the colored surface cast in shadow as the apparent color of the surface leaves illumination-dependent properties out of account.

Being more or less brightly lit is an illumination-dependent property of things. When a uniformly colored surface is unevenly illuminated (because of the position and character of the illuminant), some parts of the surface will be more brightly lit than others. The distinctive appearance of the colored surface in the brightly lit parts is unlike the distinctive appearance of the less brightly lit parts. This difference is not a difference in the apparent color of the different parts of the surface. It is the different ways the uniform color appears when more or less brightly lit. This is again analogous to Austin's own example of perceptual constancy. Understanding the difference in appearance of the differently illuminated parts as a difference in apparent color of the surface leaves illumination-dependent properties out of account.

The possession of a specular highlight is an illumination-dependent property of things. Specular highlights are particularly intense reflections of incident light that are perceiver- and illumination dependent. As we have seen, the apparent color of the specular highlight of a red car is not the apparent color of the car. At least as ordinarily understood, the apparent color of the car, the color it appears to have, is red. The apparent color of the specular highlight, in contrast, is the color of the light reflected from the car's surface---in this case, white. Understanding the color of a specular highlight as the apparent color of the surfaces of things leaves illumination-dependent properties out of account.

In being thus insensitive to the illumination-dependent properties of things, Noë's understanding of experience is, in one way, even more impoverished than Berkeley's. For Berkeley, what we strictly speaking see is confined to ``light and color''. Apparently, for Noë, we do not strictly speaking see illumination-dependent properties; rather, they come to figure in our experience thanks to our practical understanding of the sensory-motor contingencies that apparent color, in Noë's technical sense, is subject to. 

In this way, experience, as conceived by Noë is, if not an inner picture, then, in one important respect, picture-like. Consider a picture of a white wall partly cast in shadow. How are the illumination-dependent features of the scene depicted? The shadow cast upon the wall will be painted gray, and the sunlit portion of the wall will be painted a warm white. A picture may depict the color of the illuminant, but the color of the illuminant is not any of the colors distributed on its two-dimensional surface. In looking at the picture we see the color of the illumination-dependent properties \emph{in} the colors distributed on the surface of the picture. We see the colors on the surface of the picture and thereby come to experience the color the illuminant is depicted as having. Similarly, on Noë's account, we see the illumination-dependent properties of the scene \emph{in} the apparent colors distributed in the scene. We see the apparent colors of surfaces and only thereby come to experience the color of the illuminant. 

A proper understanding of color constancy calls for a more thorough-going rejection of the inner representation model.

% section vestigal_remnants_of_inner_representations (end)

\section{The Problem of Unity} % (fold)
\label{sec:the_problem_of_unity}

According to Noë, when an object looks a particular way in some circumstance of perception, there is a relational property, its \emph{look} or \emph{appearance}, that the object has in those circumstances. Apparent colors, in Noë's technical sense of the term, are among these relational properties. Noë's metaphysics of color, \emph{phenomenal objectivism}, builds upon this idea. Colors are identified with patterns of relational looks or appearances, or ``color aspect profiles'', that vary as relevant aspects of the circumstances of perception, or ``color critical conditions'', vary. For Noë, no less than for \citet{Locke:1706hc}, to \emph{be colored} is to \emph{look colored}. (For more on the connection between phenomenal objectivism and secondary qualities see \citealt{Allen:2008kx}.)

Phenomenal objectivism is \emph{phenomenalist} insofar as it involves the reductive identification of colors with color aspect profiles---colors are nothing over and above patterns of these relational looks or appearances. On this view, there is no underlying color or chromatic principle that is manifest in the way that the look of a colored thing varies with the circumstances of perception. The color of an object neither grounds nor explains, even in part, the different color appearances it has in different circumstances of perception. Phenomenal objectivism is \emph{objectivist} insofar as these relational looks or appearances are mind-independent features of the material environment. The apparent gray manifest by the white wall in shadow is a relational property that the wall may have independently of being perceived. Crucially, it is the relation between the wall and the light source and not the any relation between the wall and a perceiver that determines this relational look or appearance. So apparent colors are not in this way subjective or mind-dependent.

Phenomenal objectivism is subject to a metaphysical problem---\emph{the problem of unity}. (A parallel problem affects secondary quality accounts, see \citealt{Kalderon:2006fk}; for an application of that problem to phenomenal objectivism see \citealt{Allen:2008kx}.) Moreover, the problem of unity prevents Noë from giving an adequate account of color constancy. Recall, color constancy is explained by attributing a dual aspect in perceptual content to color experience. While we strictly speaking see only the apparent colors of things, colors figure in our experience thanks to our practical understanding of the way apparent colors vary with color critical conditions. It is the color aspect profile that remains constant or uniform despite the difference in apparent color. Unfortunately, as we will see, the problem of unity prevents color aspect profiles from playing this rôle.

Notice not every pattern of apparent color is a way of being colored. Being apparent red in one circumstance and being apparent green in another circumstance do not figure in any pattern of apparent color associated with individual colors. Joint possession of these appearances in these circumstances is not a way of being colored. (A \textsc{cd} may appear red from one angle and green from another angle, but what color is a \textsc{cd}?) So not every pattern of apparent color constitutes a color. How are we to distinguish the patterns of apparent colors that are ways of being colored from those patterns of apparent colors that are not? What unites the different appearances in different circumstances of perception as a way of being colored?

The challenge is acute since a naïve answer is precluded. The pattern of apparent colors could not be united by the color itself. To appear red in certain circumstances would be to appear the way red things appear to normal perceivers in those circumstances. So conceived, however, colors would be mind-in\-de\-pen\-dent qualities of the material environment that explain, in part, the way they appear in different circumstances of perception. (For a defense of the naïve answer see Yablo's, \citeyear{Yablo:1995fk}, discussion of singling out properties. See also \citealp{Campbell:1997dq}.) Phenomenal objectivism, however, precludes this answer. It is \emph{phenomenalist} precisely in denying that color is a property over and above the pattern of apparent colors. And if color is nothing over and above a pattern of color appearances, then the color of an object neither grounds nor explains, even in part, the different color appearances it has in different circumstances of perception.

Unfortunately, he problem of unity is inseparable from the phenomenon of color constancy. As Johnston observes: 
	\begin{quote}
		A basic phenomenological fact is that we see most of the colors of external things as ``steady'' features of those things, in the sense of features which do not alter as the light alters and as the observer changes position (this is sometimes called ``color constancy''.) A course of experience as of the steady colors is a course of experience as of light-independent and observer-independent properties, properties simply made evident to appropriately placed perceivers by adequate lighting. Contrast the highlights: a course of experience as of the highlights reveals their relational nature. They change as the observer changes position relative to the light source. They darken markedly as the light source darkens. With sufficiently dim light they disappear while the ordinary color remains. \citep[141]{Johnston:1992ck} 
	\end{quote}
Light-independent and observer-independent properties ``simply made evident to appropriately placed perceivers by adequate lighting'' are sensible qualities of the material environment presented to a perceiver's partial perspective on that environment. Suppose, with Noë, that we strictly speaking see only the apparent color of an object that can vary with the circumstances of perception. The appearance of the constant color persists through these variable appearances. What is made evident to appropriately placed perceivers by adequate lighting is what \emph{unites} these variable appearances. Thus without an answer to the problem of unity, phenomenal objectivism lacks an explanation of color constancy. But the problem of unity is insoluble. What is made evident may be what unites the pattern of appearances, but what's made evident to appropriately placed perceivers by adequate lighting is simply the color of the object. So conceived, however, colors would be mind-independent qualities of the material environment that would explain, in part, the way they appear in different circumstances of perception. This is an application of Anscombe's insight: 
	\begin{quote}
		Further, we ought to say, not: ``Being red is looking red in normal light to the normal-sighted,'' but rather ``Looking red is looking as a thing that \emph{is} red looks in normal light to the normal-sighted.'' \citep[14]{Anscombe:1981fk} 
	\end{quote}

% section the_problem_of_unity (end)

\section{The Look of Looks} % (fold)
\label{sec:the_look_of_looks}

If the reflections of the previous section are correct, it would seem that the phenomenology of color constancy require that the color of an object explains the different ways that it appears in different circumstances of perception in a manner that is inconsistent with phenomenal objectivism. However, Noë thinks he has a simple yet powerful argument that nothing like this could be true. According to Noë, colors are themselves looks or appearances, and looks or appearances are not iterative---there are no looks of looks. So there is nothing that could be the real color that grounds or explains its apparent color, the way the real shape of an object grounds or explains its apparent shape:
	\begin{quote}
		[C]olors, unlike shapes, it would seem, are themselves looks. This would seem to make apparent colors the looks of looks, a notion that is probably not coherent. The problem, at base, is this: If colors, in contrast with shapes, are ways things look, then it is not possible to explain our experience of the actual color of a thing in terms of looks, in the way that we were able to explain the experience of the actual shape of a thing in terms of our experience of how it looks (its P-shape) from here. For the way a thing looks with respect to color from here is just another experience of color. There is nothing, it would seem, that stands to color as P-properties (perspectival shape and perspectival size) stand to their corresponding properties. \citep[133]{Noe:2004fk}
	\end{quote}

Colors \emph{are} looks or appearances, on at least one reasonable construal of that claim. As \citet[109]{Strawson:1966kx} remarks, ``colours are visibilia or they are nothing''. But visibilia can themselves have looks or appearances. Among the visibilia I encounter upon entering the wedding party is Edgar's dour look, a look that looked out of place on that happy occasion. However, pointing out that looks can sometimes iterate does not entail that they always do. There may yet be a problem specific to the colors that prevents the colors, being looks, from themselves having looks or appearances.

What's the disanalogy between shapes and colors? The crucial moment in the quoted passage is this:
	\begin{quote}
		For the way a thing looks with respect to color from here is just another experience of color. \citep[133]{Noe:2004fk}
	\end{quote}
\citet{Allen:2008kx} interprets its significance as follows:
	\begin{quote}
		The thought seems to be that an apparent colour cannot stand to a ‘real’ colour in the way that apparent shape stands to ‘real’ shape, because the apparent colour and the ‘real’ colour both need to be understood with reference to our experience of the object. But, the objection continues, we cannot understand what it would be for an object to look one way in respect of colour in a particular set of perceptual circumstances whilst looking a different way in respect of colour more generally, because we cannot simultaneously experience the object as appearing two different ways with respect to colour: the circumstances in which an object is seen determines the way the object looks in respect of colour, and screens off its looking any other way. So, for instance, we cannot experience part of a wall that is in shadow as simultaneously looking grey and white, as the first look---the look in these particular circumstances, namely grey---screens off the second look---the object’s more general look, namely white.
	\end{quote}
So understood, if colors themselves had looks or appearances, then the prima facie conflict in the content of color experience would be genuine. But the prima facie conflict could not be genuine. Hence colors could not themselves have looks or appearances.

I have already argued that the prima facie conflict in perceptual content does not arise in a way that requires acknowledging the dual aspect in perceptual content. It is worth revisiting, however. The putative incoherence of a color having different looks or appearances in different circumstances of perception turns on an \emph{equivocation} \citep[though I develop the point in a different way, it is essentially due to][48]{Chisholm:1957dq}. 

The appearance of the pig startled Edgar. This statement admits of different construals on different occasions. Edgar may be startled by the wounded look of the pig. Or Edgar may be startled by an unexpected porcine presence, as when the pig interrupts the wedding party and begins to graze at the buffet table. In each case, we may say that Edgar was startled by the pig's appearance, but there is no one thing that is the appearance of the pig on each occasion. 

In the first case, the appearance of the pig is some visible aspect of the pig---a visible aspect that grounds objective similarities to other visible things. In looking wounded, the pig may, for example, look the way Bernice would look like upon hearing some terrible news. It is this visible aspect of the pig that Edgar finds startling. In the second case, the appearance of the pig is not some visible aspect of the pig, but the presentation of the pig itself. It is the presence of the pig at the wedding party that Edgar finds startling, and not its gluttonous look as it happily helps itself to the canapés.

Similarly, for color appearances. The gray appearance of the wall made Edgar melancholy. This statement admits of different construals on different occasions. Edgar may be melancholy because of the dark look of the wall. Or Edgar may be melancholy because the color scheme for the room he is decorating has unexpectedly gone awry---it is not the color he instructed the painters to paint the wall. In each case, we may say that Edgar was made melancholy by the gray appearance of the wall, but there is no one thing that is the gray appearance on each occasion.

In the first case, the gray appearance is some visible aspect of the wall---a visible aspect that grounds objective similarities to other visible things. In looking gray, the wall may, for example look the way an unshadowed gray thing would look like. It is this visible aspect that makes Edgar melancholy. Looking this way needn't mean that the wall is gray, or even looks to be gray. Think of the by now familiar white wall cast in shadow. If the circumstances of perception are within the limits of human color constancy, then the shadowed white thing looks like an unshadowed gray thing even though it does not look to be gray. It may be white, and seen to be white by Edgar, but its dark look under present conditions of illumination is sufficient to provoke gloom. In the second case, it is the presence of grayness and not the gray look of a non-gray thing that depresses Edgar---it simply doesn't fit his present decorating scheme. Here, the gray appearance of the wall is the grayness of the wall made evident in Edgar's experience of it. Edgar sees the gray of the wall. It looks to be gray.

So in the first case the gray appearance of the wall is a visible aspect of the wall that may be registered by a comparative looks statement. So understood, the wall may look, under present circumstances, the way a gray wall would look like under some other, contextually salient, circumstances. In the second case, the gray appearance is the presentation of the wall's grayness in Edgar's experience of it---gray is the color it looks to be.

In the case of the white wall cast in shadow, the prima facie conflict is that the wall appears white and that part of the wall cast in shadow appears gray, but nothing can be both uniformly white and partly gray. The sense that there is even a prima facie conflict here is the result of an equivocation. The shadowed wall may appear gray. But that only means that the white wall cast in shadow looks \emph{like} an unshadowed gray thing. Given that this is meant to be a case of color constancy, the shadowed wall looks \emph{to be} white. To generate the conflict, then, the shadowed wall would not only have to look \emph{like} a gray thing it would have to look \emph{to be} gray. But the shadowed white wall doesn't look to be gray even though it looks like an unshadowed gray thing (Hering's ringed shadow experiment established that).

% section the_look_of_looks (end)

\section{Naïve Realism and the Partiality of Perception} % (fold)
\label{sec:naïve_realism_and_the_partiality_of_perception}

Consider Hume's characterization of experience as conceived by the vulgar:
\begin{quote}
    \ldots when men follow this blind and powerful instinct of nature, they always suppose the very images, presented by the senses, to be the external objects, and never entertain any suspicion, that the one are nothing but representations of the other. This very table, which we see white, and which we feel hard, is believed to exist independent of our perception, and to be something external to our mind, which perceives it. \citep[113--4]{Hume:1740lr}
\end{quote}
According to this naïve conception, veridical experience is \emph{relational}. When I look at the tomato ripening on my window sill, among the elements of the scene that are presented to my perspective on that scene is the red of the tomato. The red of the tomato is a determinate, spatiotemporally located color instance. This determinate, spatiotemporally located color instance is a \emph{constituent} of my experience. An experience would be intrinsically connected to its subject matter since experience, so conceived, just is a perceptual presentation of that subject matter to a perceiver's perspective.

Not only is experience the presentation of elements of the scene to the perceiver's perspective on that scene, but this perspective is also \emph{partial}. The partiality of perception has recently been defended by \citet{Hilbert:1987jq}, but it has ancient roots as well---arguably, Heraclitus is an advocate \citep[see][]{Burnyeat:1979mv,Kalderon:2006tg}:
	\begin{quote}
		Our perceptions never provide us with complete information about the properties of the objects we are perceiving. They also may provide partial information about determinates lying under some determinable. Vision provides partial information about objects, not only in the sense that there are properties that are not visually accessible, but also in the sense that even properties that are visually accessible may not be completely determined by any given visual perception.
	\end{quote}
If perception provides only a partial perspective on the material environment, then that property will appear differently in different circumstances of perception. The color that the tomato ripening on my window sill appears to have will appear differently in the morning light than it will when I bring it into the kitchen for slicing.

Human color constancy is imperfect. Not only does human color vision display constancy for only of some scenes and some conditions of illumination, human color vision displays different \emph{degrees} of constancy in different kinds of scenes in different ranges of illumination. Hilbert explains how human color constancy is imperfect in a further important sense:
	\begin{quote}
		Many theories of color constancy take the form of explaining how it is that the visual system manages to extract information about the reflectance of the objects in a scene from the color signal from those objects. Since this involves separating the contributions of the reflectance and the illuminant to the color signal these theories are often characterized as ``discounting the illuminant''. Perfect color constancy in these terms would involve accurate recovery of reflectance for any scene under any lighting conditions. The perceived color of objects would be perfectly correlated with their reflecting characteristics and not vary at all with changes in the illuminant of the composition and arrangement of objects in view. This type of perfect color constancy is not possible. \citep[143]{Hilbert:2007qy}
	\end{quote}

Human color constancy is partial and variable---it holds in different degree in different kinds of scenes in different ranges of illumination. This should not be thought of as a deficit. Suppose there could be a perceiver whose perception displayed perfect color constancy in this sense. What would it be like for them to see a field of grass set against a blue summer sky? The field would appear uniformly green and the sky uniformly blue. Moreover, no difference in color appearance would differentiate any portion of the uniformly green field. The experience of the scene would be not unlike a young child's drawing of the scene. The grass would be green (and lack the golden cast that we might observe in viewing the same scene, nor would it be dappled, as we observe the scene to be, by sunlight and shadow). Children's drawings also intimate what perfect size constancy might be like---they will draw a car as larger than a man even if the car is at a great distance from the man. Just as with perfect size constancy we would lose information about distance, so with perfect color constancy we would lose information about the illuminant. So the partial and variable character of human color constancy is no deficit. And not merely because it lacks the garish character of children's crayon drawings, but because we would be insensitive to important aspects of our environment.


% section naïve_realism_and_the_partiality_of_perception (end)

% Bibligography
\bibliographystyle{plainnat} 
\bibliography{Philosophy.bib} 

\end{document}
