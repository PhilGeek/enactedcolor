%!TEX TS-program = xelatex 
%!TEX TS-options = -output-driver="xdvipdfmx -q -E"
%!TEX encoding = UTF-8 Unicode
%
%  enactedcolor
%
%  Created by Mark Eli Kalderon on 2008-07-01.
%

\documentclass[12pt]{article} 

% Definitions
\newcommand\mykeywords{color, constancy, Noë, enacted perception, perception} 
\newcommand\myauthor{Mark Eli Kalderon} 
\newcommand\mytitle{Color and the Problem of Perceptual Presence}

% Packages
\usepackage{geometry} \geometry{a4paper} 
\usepackage{url}
\usepackage{pdfsync} 
\usepackage{txfonts}
\usepackage{color}
\definecolor{gray}{rgb}{0.459,0.438,0.471}
% \usepackage{setspace}
% \doublespace % Uncomment for doublespacing if necessary
% \usepackage{epigraph} % optional

% XeTeX
\usepackage[cm-default]{fontspec}
\usepackage{xltxtra,xunicode}
\defaultfontfeatures{Scale=MatchLowercase,Mapping=tex-text}
\setmainfont{Hoefler Text}
\setsansfont{Gill Sans}
\setmonofont{Inconsolata}

% Version Control Information
\immediate\write18{sh ./vc}
\input{vc}


% Section Formatting
\usepackage[]{titlesec}
\titleformat{\section}[hang]{\fontsize{14}{14}\scshape}{\S{\thesection}}{.5em}{}{}
\titleformat{\subsection}[hang]{\fontsize{12}{12}\scshape}{\S{\thesubsection}}{.5em}{}{}
\titleformat{\subsubsection}[hang]{\fontsize{12}{12}\scshape}{\S{\thesubsubsection}}{.5em}{}{}

% Headers and Footers
\usepackage{fancyhdr}
\pagestyle{fancy}
\pagenumbering{arabic}
\lhead{\thepage}
\chead{}
\rhead{\itshape{\nouppercase{\leftmark}}}
\lfoot{\tiny{Revision: \VCRevision}}
\cfoot{\tiny{Author: \VCAuthor}}
\rfoot{\tiny{Date: \VCDateISO}}

% Bibliography
\usepackage[round]{natbib} 

% Title Information
\title{\mytitle}% Thanks optional \thanks{}
\author{\myauthor} 
% \date{} % Leave blank for no date, comment out for most recent date

% PDF Stuff
\usepackage[plainpages=false, pdfpagelabels, bookmarksnumbered, backref, pdftitle={\mytitle}, pagebackref, pdfauthor={\myauthor}, pdfkeywords={\mykeywords}, xetex, dvipdfmx, colorlinks=true, citecolor=gray, linkcolor=gray, urlcolor=gray]{hyperref} 

%%% BEGIN DOCUMENT
\begin{document}

% Title Page
\maketitle
% \begin{abstract} % optional
% \end{abstract} 
\vskip 2em \hrule height 0.4pt \vskip 2em
% \epigraph{text of epigraph}{\textsc{author of epigraph}} % optional; make sure to uncomment \usepackage{epigraph}

% Layout Settings
\setlength{\parindent}{1em}

% Main Content

\section{The Problem of Perceptual Presence}\label{sec:the_problem_of_perceptual_presence} % (fold)

According to Noë, we experience more than we ``strictly speaking'' see. 

When I look at the tomato ripening on my window sill, I experience a voluminous whole even though I strictly speaking see only the frontside of the tomato \citep[76]{Noe:2004fk}. When I look at Ricca's cat as he passes behind the leg of the kitchen table, I experience the whole cat even though, I strictly speaking see only those parts of the cat not obscured by the table leg \citep[60]{Noe:2004fk}. When I look at the plate on that table, I experience the plate as circular even though, from my current vantage point, I strictly speaking see only its elliptical look \citep[78--79]{Noe:2004fk}. When I look up at the unevenly illuminated wall, I experience the wall as uniformly white even though I strictly speaking see only gradations of gray \citep[127]{Noe:2004fk}. 

% Insofar as we do not strictly speaking see these things, they are absent; insofar as they nevertheless figure in our experience, they are present. 

What is it to experience voluminous wholes, real shapes, and uniform colors? It is to have a practical understanding of how the look or appearance of things varies with the conditions under which they are perceived:
\begin{quote}
	The experience of shape depends on our implicit grasp of the way perspectival shape varies as we move in respect to an object. We don’t have names for every aspect we encounter, but we have a grip on the way aspects vary. This grip is, in effect, our grasp of what it is for something to be presented as cubical, or spherical. It is much harder to make out what our grasp on the form of a shoulder, or a human jaw, or a hip, consists in. But there is no reason, in principle, why it cannot consist in something very much like this \ldots\. Similar kinds of considerations \ldots\ go for color: Our grasp of color depends on our implicit mastery of the way appearances change as color critical conditions change. \citep[198--199]{Noe:2004fk}
\end{quote}
This analysis is developed, refined, and extended (for example, to account for Molyneux's Question) throughout \emph{Action in perception}. I will have more to say about it presently. But for now, I want to consider what could motivate this analysis in the first place. Why think that we experience more than we strictly speaking see?



% section the_problem_of_perceptual_presence (end)

% Bibligography
\bibliographystyle{plainnat} 
\bibliography{Philosophy.bib} 

\end{document}
