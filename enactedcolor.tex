%!TEX TS-program = xelatex 
%!TEX TS-options = -output-driver="xdvipdfmx -q -E"
%!TEX encoding = UTF-8 Unicode
%
%  enactedcolor
%
%  Created by Mark Eli Kalderon on 2008-07-01.
%

\documentclass[12pt]{article} 

% Definitions
\newcommand\mykeywords{color, constancy, Noë, enacted perception, perception} 
\newcommand\myauthor{Mark Eli Kalderon} 
\newcommand\mytitle{Color and the Problem of Perceptual Presence}

% Packages
\usepackage{geometry} \geometry{a4paper} 
\usepackage{url}
\usepackage{pdfsync} 
\usepackage{txfonts}
\usepackage{color}
\definecolor{gray}{rgb}{0.459,0.438,0.471}
% \usepackage{setspace}
% \doublespace % Uncomment for doublespacing if necessary
% \usepackage{epigraph} % optional

% XeTeX
\usepackage[cm-default]{fontspec}
\usepackage{xltxtra,xunicode}
\defaultfontfeatures{Scale=MatchLowercase,Mapping=tex-text}
\setmainfont{Hoefler Text}
\setsansfont{Gill Sans}
\setmonofont{Inconsolata}

% Version Control Information
\immediate\write18{sh ./vc}
\input{vc}


% Section Formatting
\usepackage[]{titlesec}
\titleformat{\section}[hang]{\fontsize{14}{14}\scshape}{\S{\thesection}}{.5em}{}{}
\titleformat{\subsection}[hang]{\fontsize{12}{12}\scshape}{\S{\thesubsection}}{.5em}{}{}
\titleformat{\subsubsection}[hang]{\fontsize{12}{12}\scshape}{\S{\thesubsubsection}}{.5em}{}{}

% Headers and Footers
\usepackage{fancyhdr}
\pagestyle{fancy}
\pagenumbering{arabic}
\lhead{\thepage}
\chead{}
\rhead{\itshape{\nouppercase{\leftmark}}}
\lfoot{\tiny{Revision: \VCRevision}}
\cfoot{\tiny{Author: \VCAuthor}}
\rfoot{\tiny{Date: \VCDateISO}}

% Bibliography
\usepackage[round]{natbib} 

% Title Information
\title{\mytitle}% Thanks optional \thanks{}
\author{\myauthor} 
% \date{} % Leave blank for no date, comment out for most recent date

% PDF Stuff
\usepackage[plainpages=false, pdfpagelabels, bookmarksnumbered, backref, pdftitle={\mytitle}, pagebackref, pdfauthor={\myauthor}, pdfkeywords={\mykeywords}, xetex, dvipdfmx, colorlinks=true, citecolor=gray, linkcolor=gray, urlcolor=gray]{hyperref} 

%%% BEGIN DOCUMENT
\begin{document}

% Title Page
\maketitle
% \begin{abstract} % optional
% \end{abstract} 
\vskip 2em \hrule height 0.4pt \vskip 2em
% \epigraph{text of epigraph}{\textsc{author of epigraph}} % optional; make sure to uncomment \usepackage{epigraph}

% Layout Settings
\setlength{\parindent}{1em}

% Main Content

\section{Perceptual Constancy}\label{sec:perceptual_constancy} % (fold)

Very often, objects in the scene before us are somehow perceived to be constant or uniform or unchanging in color, shape, size, or position, even while they their appearance with respect to these features somehow changes. This is a familiar and pervasive fact about perception, even if it is notoriously difficult to describe accurately let alone adequately account for. These difficulties are not unrelated---how we are inclined to \emph{describe} the phenomenology of perceptual constancy will affect how we are inclined to \emph{account} for it.

% section perceptual_constancy (end)

\section{The Problem of Perceptual Presence}\label{sec:the_problem_of_perceptual_presence} % (fold)

According to Noë, we experience more than we ``strictly speaking'' see. 

When I look at the tomato ripening on my window sill, I experience a voluminous whole even though I strictly speaking see only the frontside of the tomato \citep[76]{Noe:2004fk}. When I look at Ricca's cat as he passes behind the leg of the kitchen table, I experience the whole cat even though, I strictly speaking see only those parts of the cat not obscured by the table leg \citep[60]{Noe:2004fk}. When I look at the plate on that table, I experience the plate as circular even though, from my current vantage point, I strictly speaking see only its elliptical look \citep[78--79]{Noe:2004fk}. When I look up at the unevenly illuminated wall, I experience the wall as uniformly white even though I strictly speaking see only gradations of gray \citep[127]{Noe:2004fk}. 

% Insofar as we do not strictly speaking see these things, they are absent; insofar as they nevertheless figure in our experience, they are present. 

What is it to experience voluminous wholes, real shapes, and uniform colors? It is to have a practical understanding of how the look or appearance of things varies with the conditions under which they are perceived:
\begin{quote}
	The experience of shape depends on our implicit grasp of the way perspectival shape varies as we move in respect to an object. We don’t have names for every aspect we encounter, but we have a grip on the way aspects vary. This grip is, in effect, our grasp of what it is for something to be presented as cubical, or spherical. It is much harder to make out what our grasp on the form of a shoulder, or a human jaw, or a hip, consists in. But there is no reason, in principle, why it cannot consist in something very much like this \ldots\. Similar kinds of considerations \ldots\ go for color: Our grasp of color depends on our implicit mastery of the way appearances change as color critical conditions change. \citep[198--199]{Noe:2004fk}
\end{quote}
This analysis is developed, refined, and extended (for example, to account for Molyneux's Question) throughout \emph{Action in perception}. I will have more to say about it presently. But for now, I want to consider what could motivate this analysis in the first place. Why think that we experience more than we strictly speaking see?

Sometimes Noë seems motivated by a dogmatic adherence to an alleged phenomenological datum---that it is just \emph{obvious} that we do not see the tomato that we experience. But if it is obvious, then why do so many theorists of perception disagree? It can't be that they are \emph{all} misled by the picture of experience as an inner representation \citep[chapter 2]{Noe:2004fk}, for not all who disagree accept that picture---naïve realists and disjunctivists maintain that veridical perception is relational and hence nonrepresentational and yet claim that among the objects that sight affords awareness of are voluminous wholes such as garden variety tomatoes. Nor is it scarcely plausible that a plurality of alien mentalities pervade among theorists of perception. It is not as if we may conclude as \citet[69]{Sartwell:1995ve}, wryly puts it, that ``some people \ldots\ are appeared to pigly, other[s] have pigs thrust upon them''. 

I believe that if we attend closely to Noë's description of perceptual phenomena we can discern a line of thought that can be reconstructed as an argument. When we do, the argument turns to be a variant of an ancient argument, \emph{the argument from conflicting appearances}.

% section the_problem_of_perceptual_presence (end)

\section{Conflicting Appearances}\label{sec:conflicting_appearances} % (fold)

Some evidence for this suggestion is provided by Noë himself:
\begin{quote}
	When you look at the wall, you see its uniform color \emph{in} its evident variation in color across its surface. When you look at a circular plate, held up at an angle, you experience its circularity \emph{in} its merely elliptical shape. When you look at a tomato, you experience it as full-bodied and three dimensional even though you don't see its sides or back; you experience its three-dimensionality in its visible parts. Part of what makes the study of perception so difficult is the necessity of acknowledging not only this dual aspect in perceptual content, \emph{but the prima facie conflict in perceptual content}. [my emphasis]
\end{quote}
Indeed, it is plausible that Noë thinks that the necessity in acknowledging the dual aspect of perceptual content is due to the way it resolves the prima facie conflict in perceptual content. But I am getting ahead of myself.

Wherein is the alleged prima facie conflict in perceptual content?

Since our topic is perceptual constancy, and color constancy in particular, let's focus on these cases.


% section conflicting_appearances (end)

% Bibligography
\bibliographystyle{plainnat} 
\bibliography{Philosophy.bib} 

\end{document}
