%!TEX TS-program = xelatex 
%!TEX TS-options = -output-driver="xdvipdfmx -q -E"
%!TEX encoding = UTF-8 Unicode
%
%  enactedcolor
%
%  Created by Mark Eli Kalderon on 2008-07-01.
%

\documentclass[12pt]{article} 

% Definitions
\newcommand\mykeywords{color, constancy, Noë, enacted perception, perception} 
\newcommand\myauthor{Mark Eli Kalderon} 
\newcommand\mytitle{Color and the Problem of Perceptual Presence}

% Packages
\usepackage{geometry} \geometry{a4paper} 
\usepackage{url}
\usepackage{pdfsync} 
\usepackage{txfonts}
\usepackage{graphicx}
\usepackage{color}
\definecolor{gray}{rgb}{0.459,0.438,0.471}
% \usepackage{setspace}
% \doublespace % Uncomment for doublespacing if necessary
% \usepackage{epigraph} % optional

% XeTeX
\usepackage[cm-default]{fontspec}
\usepackage{xltxtra,xunicode}
\defaultfontfeatures{Scale=MatchLowercase,Mapping=tex-text}
\setmainfont{Hoefler Text}
\setsansfont{Gill Sans}
\setmonofont{Inconsolata}

% % Version Control Information
% \immediate\write18{sh ./vc}
% \input{vc}


% Section Formatting
\usepackage[]{titlesec}
\titleformat{\section}[hang]{\fontsize{14}{14}\scshape}{\S{\thesection}}{.5em}{}{}
\titleformat{\subsection}[hang]{\fontsize{12}{12}\scshape}{\S{\thesubsection}}{.5em}{}{}
\titleformat{\subsubsection}[hang]{\fontsize{12}{12}\scshape}{\S{\thesubsubsection}}{.5em}{}{}

% Headers and Footers
\usepackage{fancyhdr}
\pagestyle{fancy}
\pagenumbering{arabic}
\lhead{\thepage}
\chead{}
\rhead{\itshape{\nouppercase{\leftmark}}}
% \lfoot{\tiny{Revision: \VCRevision}}
% \cfoot{\tiny{Author: \VCAuthor}}
% \rfoot{\tiny{Date: \VCDateISO}}

% Bibliography
\usepackage[round]{natbib} 

% Title Information
\title{\mytitle}% Thanks optional \thanks{}
\author{\myauthor} 
% \date{} % Leave blank for no date, comment out for most recent date

% PDF Stuff
\usepackage[plainpages=false, pdfpagelabels, bookmarksnumbered, backref, pdftitle={\mytitle}, pagebackref, pdfauthor={\myauthor}, pdfkeywords={\mykeywords}, xetex, dvipdfmx, colorlinks=true, citecolor=gray, linkcolor=gray, urlcolor=gray]{hyperref} 

%%% BEGIN DOCUMENT
\begin{document}

% Title Page
\maketitle
% \begin{abstract} % optional
% \end{abstract} 
\vskip 2em \hrule height 0.4pt \vskip 2em
% \epigraph{text of epigraph}{\textsc{author of epigraph}} % optional; make sure to uncomment \usepackage{epigraph}

% Layout Settings
\setlength{\parindent}{1em}

% Main Content

\section{Perceptual Constancy}\label{sec:perceptual_constancy} % (fold)

Very often, objects in the scene before us are somehow perceived to be constant or uniform or unchanging in color, shape, size, or position, even while they their appearance with respect to these features somehow changes. This is a familiar and pervasive fact about perception, even if it is notoriously difficult to describe accurately let alone adequately account for. These difficulties are not unrelated---how we are inclined to \emph{describe} the phenomenology of perceptual constancy will affect how we are inclined to \emph{account} for it.

\citet{Noe:2004fk} believes that the facts of color constancy, that the color of a thing appears to persist unaltered though it presents distinct appearances, speaks in favor of a particular view about color experience and a particular view about the metaphysics of color. Though we experience the persistent unaltered color, we strictly speaking see only the variable \emph{look} or \emph{appearance}. Moreover, colors are nothing over and above the patterns of variable looks or appearances.

Though there is much to admire in Noë's account of perception---particularly noteworthy is his resistance to any conception of experience as a form of inner representation, nevertheless, his views about color experience and the metaphysics of color are inadequate. Specifically, they fail to account for the very phenomena that motivates them. The root of this difficulty is that Noë has misdescribed color constancy at the outset. A better description of the phenomenology motivates a naïve realism about color experience and the metaphysics of color, a naïve realism that retains many of the virtues that Noë claims for his own account \citep[see][for similar suggestions]{Allen:2008kx,Campbell:2008nx}.

% section perceptual_constancy (end)

\section{The Problem of Perceptual Presence}\label{sec:the_problem_of_perceptual_presence} % (fold)

According to Noë, we experience more than we ``strictly speaking'' see. 

When I look at the tomato ripening on my window sill, I experience a voluminous whole even though I strictly speaking see only the frontside of the tomato \citep[76]{Noe:2004fk}. When I look at Ricca's cat, Spock, as he passes behind the leg of the kitchen table, I experience the whole cat even though, I strictly speaking see only those parts of Spock not obscured by the table leg \citep[60]{Noe:2004fk}. The same goes for constancy phenomena. When I look at the plate on the table, I experience the plate as circular even though, from my current vantage point, I strictly speaking see only its elliptical look \citep[78--79]{Noe:2004fk}. And when I look up at the unevenly illuminated wall, I experience the wall as uniformly white even though I strictly speaking see only gradations of gray \citep[127]{Noe:2004fk}. 

% Insofar as we do not strictly speaking see these things, they are absent; insofar as they nevertheless figure in our experience, they are present. 

What is it to experience voluminous wholes, real shapes, and uniform colors? It is to have a practical understanding of how the look or appearance of things varies with the conditions under which they are perceived:
	\begin{quote}
		The experience of shape depends on our implicit grasp of the way perspectival shape varies as we move in respect to an object. We don’t have names for every aspect we encounter, but we have a grip on the way aspects vary. This grip is, in effect, our grasp of what it is for something to be presented as cubical, or spherical. It is much harder to make out what our grasp on the form of a shoulder, or a human jaw, or a hip, consists in. But there is no reason, in principle, why it cannot consist in something very much like this \ldots\ Similar kinds of considerations \ldots\ go for color: Our grasp of color depends on our implicit mastery of the way appearances change as color critical conditions change. \citep[198--199]{Noe:2004fk}
	\end{quote}
This analysis is developed, refined, and extended (for example, to account for Molyneux's Question) throughout \emph{Action in perception}. I will have more to say about it presently. But for now, I want to consider what could motivate this analysis in the first place. Why think that we experience more than we strictly speaking see?

Sometimes Noë seems motivated by a dogmatic adherence to an alleged phenomenological datum---that it is just \emph{obvious} that we do not see the tomato that we experience. But if it is obvious, then why do so many theorists of perception disagree? It can't be that they are \emph{all} misled by the picture of experience as an inner representation \citep[chapter 2]{Noe:2004fk}, for not all who disagree accept that picture---naïve realists and disjunctivists maintain that veridical perception is relational and hence nonrepresentational and yet claim that among the objects that sight affords awareness of are voluminous wholes such as garden variety tomatoes. Nor is it scarcely plausible that a plurality of alien mentalities pervade among theorists of perception. It is not as if we may conclude, as \citet[69]{Sartwell:1995ve} wryly puts it, that ``some people \ldots\ are appeared to pigly, [while] other[s] have pigs thrust upon them''. 

I believe that if we attend closely to Noë's description of perceptual phenomena we can discern a line of thought that can be reconstructed as an argument. When we do, the argument turns to be a variant of an ancient argument, \emph{the argument from conflicting appearances}.

% section the_problem_of_perceptual_presence (end)

\section{Conflicting Appearances and the Conflict Within Appearance}\label{sec:conflicting_appearances} % (fold)

Some evidence for this suggestion is provided by Noë himself:
	\begin{quote}
		When you look at the wall, you see its uniform color \emph{in} its evident variation in color across its surface. When you look at a circular plate, held up at an angle, you experience its circularity \emph{in} its merely elliptical shape. When you look at a tomato, you experience it as full-bodied and three dimensional even though you don't see its sides or back; you experience its three-dimensionality in its visible parts. Part of what makes the study of perception so difficult is the necessity of acknowledging not only this dual aspect in perceptual content, \emph{but the prima facie conflict in perceptual content}. [my emphasis]
	\end{quote}
Indeed, it is plausible that Noë thinks that the necessity in acknowledging the dual aspect of perceptual content is due to the way it resolves the prima facie conflict in perceptual content.

Wherein is the alleged prima facie conflict in perceptual content?

Since our topic is perceptual constancy, and color constancy in particular, let's focus on these cases.

The character of our color experience can vary as the circumstances of perception under which we view a colored object varies. Our experience of a colored object can vary as the object's position relative to the illuminant varies, as when that object is now more, now less directly illuminated \citep[125]{Noe:2004fk}. Our experience of a colored object varies as the nature of the illuminant changes, as when we view that object now in artificial illumination, now in natural daylight \citep[125]{Noe:2004fk}. Our experience of a colored object can vary as we change our position to the object while it's position relative to the illuminant remains constant, as when we experience shifting shadows and specular highlights as we mover around a stable object \citep[125]{Noe:2004fk}. Our experience of a colored object can vary as the color of the surrounding scene varies due to color contrast effects \citep[125--127]{Noe:2004fk}---effects systematically explored in the paintings of \citet[]{Albers:1963gf}. It is important to emphasize, as \citet[62-63]{Cohen:2008hc} observes, that the variable appearance of a uniformly colored object can be presented in a single experience of that object as well as in distinct experiences of that object in distinct circumstances of perception, as when the uniformly colored object is viewed under uneven illumination. Color constancy can be \emph{simultaneous} as well as \emph{successive}.

Roughly speaking, the prima facie conflict in perceptual content is the appearance of an uniform color despite the variable color appearances:
\begin{quote}
	Despite these changes in apparent color, and despite specular colors, perceivers are usually able to recognize, say, that the car is a uniform and unchanging red. \ldots\ We see the uniformity despite, or behind or beneath (as it were), the variable appearance. We do not confuse changes in the apparent color as color critical conditions change with changes in the underlying actual color. \ldots\ We experience color as that which is, in a wide range of cases, \emph{invariant} amid the apparent variation. \citep[127]{Noe:2004fk}
\end{quote}

It is important to emphasize that this conflict \emph{could only be prima facie}. Consider cases of simultaneous color constancy, as when an uniformly colored surface is unevenly illuminated. If the conflict were genuine, then our experience of the uniform color despite its variable appearance would be like an experience of an impossible scene, such as Penrose's \citeyearpar{Penrose:1958kx} impossible triangle, or a scene depicted by an Escher drawing (see \autoref{fig:triangle}).
	\begin{figure}[htbp]
		\centering
			\includegraphics[scale=1]{triangle.jpg}
		\caption{Impossible Triangle}
		\label{fig:triangle}
	\end{figure}
However, in cases of simultaneous color constancy, our experience of the invariant color amid the variable appearances is not incoherent in this way, nor does it have the requisite tension that results from such incoherence. Nor does successive color constancy give rise to any visual puzzlement. And that means that the apparent conflict within or between experience could only be apparent. The conflict arises solely in how we might be tempted to describe the phenomenology of stability and flux. 

What, then, more precisely, is the prima facie conflict within or between color experiences? 

A closer look at some of Noë's examples is revealing:
	\begin{quote}
		For example, suppose you enter a room and see that the wall is a uniform shade of white. You also see that the wall is brighter here, where it falls in direct sunlight, than it is there, where it falls in shadow. Differences in brightness, however, mean differences in color. You see the uniformity of color despite the evident nonuniformity of different parts of the wall's surface \citep[127]{Noe:2004fk}
	\end{quote}
and again:
	\begin{quote}
		Crucially, we can experience the wall as uniform in color \emph{and} as differently colored across is surface. \citep[129]{Noe:2004fk} Just as we can see that the plate looks circular \emph{and} elliptical, so we can see the color is uniform \emph{and} variable.
	\end{quote}

As shadows are unnecessary for a difference in incident illumination, there is an additional case---a white wall that is unevenly illuminated by a lamp, say, off to one side of it. Presumably, Noë's description of this case would be parallel---a difference in brightness makes for a difference in color, and so the wall appears uniformly colored and variably colored.

Specular highlights are a third case:
	\begin{quote}
		For example, the specular highlights on the surface of a clean, new automobile vary as viewing geometry changes \ldots\ As you move in relation to the car, or as it moves in relation to you, the apparent color of the car's surface may visibly change.
	\end{quote}
And yet the color of the car appears uniform and unaltered with the change in viewing geometry.

The prima facie conflict, then, is this: The wall appears uniformly white and yet parts of the wall appear gray. The car appears uniformly red and yet parts of the car appear white. But nothing can have a visible surface that is uniformly white and partly gray and nothing can have a visible surface that is uniformly red and partly white. So we have a prima facie conflict in perceptual content.

This prima facie conflict in perceptual content grounds the necessity in acknowledging the dual aspect of perceptual content. The attribution of the dual aspect is meant to resolve this prima facie conflict. On the one hand, what we strictly speaking see are \emph{apparent colors}. The unevenly illuminated white wall varies in apparent color. (This is evidently a technical term since it departs from ordinary usage. Ordinarily, the apparent color of an object is the color it appears to have. On this, ordinary, understanding of the phrase, the apparent color of the wall is white, even where it is shadowed---so long as the difference in illumination is within the bounds of human color constancy). Apparent colors, for Noë, are objective if relational properties of things. A white wall in shadow has the relational property, being gray in shadow. On the other hand, while we strictly speaking see apparent colors, we do not strictly speaking see colors, though we experience them thanks to our practical understanding of how apparent colors change in different circumstances of perception, or ``color critical'' conditions. Colors themselves are not objective if relational properties of things, but are rather patterns of these relational looks or appearances, or ``color aspect profiles''. While nothing can have a visible surface that is uniformly white and partly gray, there is nothing inconsistent with something having a visible surface that is uniformly white a portion of which has the relational property apparent gray. Indeed part of what it is for the wall to be white is for it to manifest apparent grayness when in shadow.

% section conflicting_appearances (end)

\section{The Phenomenology of Stability and Flux}\label{sec:stability_and_flux} % (fold)

We will examine the merits of this proposal in sequel. For now, we will consider only whether Noë has adequately described the prima facie conflict that motivates it. If no conflict is generated from an adequare description of the phenomenology of stability and flux, then acknowledging the dual aspect of perceptual content is unnecessary.

\subsection{Shadows} % (fold)
\label{sub:shadows}

Let's begin with the shadowed white wall. That part of the white wall cast in shadow appears differently from the part directly illuminated, yet all of the parts appear to be uniformly white. How are we to characterize the difference? Though the parts are uniformly colored they differ in appearance. Moreover this, Noë insists, is a chromatic difference. The apparent color of the shadowed wall is some shade of gray. Notice this is what is required to generate the prima facie conflict in this case---that nothing can be uniformly white and partly gray. Noë offers two reasons for thinking that the shadowed white wall appears gray. Neither are convincing.

First, Noë appeals to the difference in brightness between the shadowed parts and the parts that are directly illuminated:
	\begin{quote}
		You also see that the wall is brighter here, where it falls in direct sunlight, than it is there, where it falls in shadow. Differences in brightness, however, mean differences in color. You see the uniformity of color despite the evident non-uniformity of different parts of the wall's surface. \citep[127]{Noe:2004fk}
	\end{quote}
The reasoning, here, can seem to turn on an equivocation. It is true that the wall is brighter where it is directly illuminated where ``brightness'', here, means the intensity of the light reflected by the white wall in direct sunlight. However, Noë needs to understand ``brightness'' as a dimension of color similarity and difference, if he is to argue from a difference in brightness to a difference in color. Unfortunately these distinct senses of brightness can come apart. Consider viewing a page of black print on white paper, first indoors under artificial illumination, then outside in natural daylight. The intensity of the light reflected by the white area of the page indoors is approximately the same as the intensity of the light reflected by the black print in sunlight \citep[199]{Peter-K:1996th}. Despite being equally bright, in the sense that the reflected light is equally intense, the apparent colors differ in chromatic brightness---the surrounding white when viewed indoors seems brighter than the black print when viewed in daylight.

Perhaps, Noë thinks that it is evident that wall in direct sunlight is brighter not only in intensity of reflected light, but also chromatically brighter, at least with respect to its apparent color. Is this really evident? He offers the following consideration:
	\begin{quote}
		Consider that although we can perceive a wall that is illuminated unevenly as uniform in color, it is also the case that when a wall is in this way illuminated unevenly, it is also visibly different with respect to color across its surface. For example to match the color of different parts of the wall, you would need different color chips. \citep[128]{Noe:2004fk}
	\end{quote}
Since we can match a gray chip to how the shadowed wall appears, the apparent color of the shadowed wall is gray. 

But what kind of a matching task are we meant to be engaged in? Suppose you are looking at the wall in question and you interpose the chip so that their apparent color may be compared. You might say that the wall looks gray, but it would wrong to conclude that the shadowed wall and the chip share the same appareance for their would remain a visual difference. Austin makes the point this way:
	\begin{quote}
		Again, it is simply not true to say that seeing a bright green after-image against a white wall is exactly like seeing a bright green patch actually on the wall; or that seeing a white wall through blue spectacles is exactly like seeing a blue wall; or that seeing pink rats in D.T.s is exactly like really seeing pink rats; or (once again) that seeing a stick refracted in water is exactly like seeing a bent stick. In all these cases we may \emph{say} the same things (`It looks blue', `It looks bent', \&c.), but this is no reason at all for denying the obvious fact that the `experiences' are different. \citep[49]{Austin:1962lr}
	\end{quote}

A straight stick in water and a bent stick may each look bent. But the straight stick in water looks different from a bent stick in at least one important respect---the appearance of an intervening medium through which the stick is viewed. There is a parallel in the present case. The part of the wall cast in shadow and the chip may each look gray. But the shadowed wall and the chip look different in at least one important respect---the appearance of the interviewing medium. Whereas the intervening medium in the case of the stick is water, the intervening medium in the case of the wall is, of course, the shadow. Not only do we see the colors of things, but we also see the way they are illuminated \citep[see][]{Hilbert:2007qy}.

Noë evidently shares the intuition with \citet[88]{Chalmers:2006kx}, that a shadowed white thing looks the same as an unshadowed gray thing. While we may grant that a shadowed white thing looks like and unshadowed gray thing, it is controversial that a shadowed white thing looks exactly like an unsahdowed gray thing. (This is an application of Austin's point which Noë, \citeyear[80]{Noe:2004fk}, notes, but makes nothing of, nor notices its present application.) Indeed there is a demonstrable difference. Consider Hering's \citeyearpar[8]{Hering:1920ty} ringed shadow experiment. When a shadow is cast on a surface you see the shadow as ``an incidental darkness that lies on the [surface]''. But the darkness is not seen as a property of the surface. Suppose we draw a black line around the shadow that completely obscures its penumbra. The darkness no longer appears to lie on the surface, but rather appears to be a property of the surface. Then, and only then, does the shadowed white surface appear gray. A shadowed white thing, with an unobscured penumbra, does not appear gray, though it may, in some respects, look like an unshadowed gray thing. Indeed, Hilbert offers the following explanation of this similarity:
	\begin{quote}
		A gray surface is one that is one whose reflectance is non-selective, i.e. fairly constant across the visible spectrum, and intermediate, i.e. lower than white but higher than black. Shading is a non-selective, partial decrease in the amount of light falling on a surface. If the visual system were to represent surface grays in terms of proportions of the best white and to represent illuminants in terms of proportions of the average illumination for the scene then there would be a formal analogy between the representation of gray surfaces and shaded surfaces. In addition, there is a similarity in the content of the representations in that both are concerned with the object’s relation to the light in its environment. In other words, a gray surface is one that reflects significantly less than all the available light, while a shaded surface is one that is illuminated by significantly less than all the available light. This similarity in content may be enough to account for the similarity in phenomenology. \citep[xx]{Hilbert:2007qy}
	\end{quote}
If we see, not only the color of an object, but also the way it is illuminated, then this explanation of the similarity is consistent with the demonstrable difference.

The problem, then, is this: The case of the shadowed wall, properly described, gives rise to no prima facie conflict in perceptual content. The prima facie conflict was meant to be that the wall appears uniformly white and partly gray. But is simply not true that the part of the wall cast in shadow appears gray. It may look, in some respects, like an unshadowed gray thing, but a visual difference remains. And with no prima facie conflict in perceptual content, acknowledging the dual aspect of perceptual content is unnecessary.

% subsection shadows (end)

\subsection{Specular Highlights} % (fold)
\label{sub:specular_highlights}

What about cases of perceptual constancy involving specular highlights? In \emph{The Problems of Philosophy}, Russell writes:
	\begin{quote}
		Although I believe that the table is ``really'' of the same colour all over, the parts that reflect the light look much brighter than the other parts, and some parts look white because of the reflected light. \citep[2]{Russell:1912uq}
	\end{quote}
Specular highlights may look brighter than the rest of the surface in the sense of the intensity of the light reflected. Specular highlights may be white set against an expanse of red. But does the part of the surface corresponding to the specular highlight itself look brighter than the rest of the surface? Is it even seen? Or is that part of the surface rather \emph{occluded} by a white specular highlight. The whiteness would then be the color of the reflected light, not, as Russell suggests, the surface. There is an analogous case: The surface of a mirror seen under intense illumination may be obscured by the glare (as when a distant compatriot signals with a mirror in the desert). If the white of the specular highlight occludes the red of the surface, then the part of the surface corresponding to the specular highlight does not appear white. It doesn't even appear!

Noë is not misled the way Russell is. He writes:
	\begin{quote}
		Speuclar highlights are frequently the color of the incident light itself, reflecting white in the sunshine, and also the colors of, for example, street lights. \citep[125]{Noe:2004fk}
	\end{quote}
However, he continues:
	\begin{quote}
		As you move in relation to the car, or as it moves in relation to you, \emph{the apparent color of the car's surface may visibly change}. [my emphasis] \citep[125]{Noe:2004fk}
	\end{quote}
Recall, as ordinarily understood, the apparent color of a thing is the color that if appears to have. If the car is red and in plain view, then the aparent color of the car is red despite its variable appearance due to shadows, reflections, and specular highlights. So, as ordinarily understood, it is simply not true that the apparent color of the car's surface may visibly change as you move in relation to the car. To claim otherwise would be to claim that the car appears to change color as you change your relative position. This, in turn, would involve Russell's mistake of counting the color of the specular highlight as the color of the car. 

Perhaps Noë has in mind the technical sense of ``apparent color'', where apparent colors are objective if relational looks or appearances. This would fit well the phenomena of specular highlights. As \citet[141]{Johnston:1992ck} observes that specular highlights ``reveal their relational nature'':
	\begin{quote}
		They change as the observer changes position relative to the light source. They darken markedly as the light source darkens. With sufficiently dim light they disappear while the ordinary color remain.
	\end{quote}
Specular hightlights are relational; they are observer- and illumination-dependent properties of things.

	These two interpretations of ``apparent color'' form the basis of a dilemma:
	\begin{itemize}
		\item If ``the apparent color'' of the car is ordinarily understood, the the color of the car appears to change as you change your relative position to it. This would generate the prima facie conflict, but it would also misdescribe the phenomenology. 
		\item If ``the apparent color'' of the car is understood in the technical sense, then it doesn't misdescribe the phenomenology in the same way, but no prima facie conflict is generated. 
	\end{itemize}
And if no prima facie conflict is generated, with what right does Noë invoke this aspect of the dual content of perceptual content? It is question-begging to appeal to objective if relational looks or appearances at this point in the discussion.

Perhaps the prima facie conflict in perceptual content is generated in another way. There is a sense in which the case of specular highlights is analagous to the case of seeing voluminous wholes. If the white is the color of the reflected light, then the color of the surface corresponding to the specular highlight is occluded. We don't see the color of that part of surface any more than we see the backside of a tomato. 

When I look at the tomato ripening on my window sill, of all the parts of the tomato that I see, I see only the frontside of the tomato. Its backside and interior are hidden from me, at least given my present vantage point and the present, unsliced, state of the tomato. Not only do I see the frontside of the tomato, I experience the tomato itself as a voluminous whole. As noted earlier, Noë maintains that while I strictly see only the frontside of the tomato I experience the tomato as a voluminous whole thanks to my practical understanding of how the look or appearance of the tomato will vary as the conditions under which it is perceived changes. That is the dual aspect of the content of my perception of the tomato. What is the prima facie conflict that grounds the necessity of acknowleding the dual aspect of this perceptual content?

Of all the parts ot the tomato that I see, I see only the frontside of the tomato. This is an instance of a familiar fact about the visibility of opaque objects. I also see the voluminous whole. \emph{So far there is no conflict}. In order to generate the prima facie conflict in perceptual content, one would need to assume, in addition, that one can only see a whole by seeing all of its parts. This is Chisholm's \citeyearpar[xx]{Chisholm:1957dq} diagnosis of why some, such as Moore, have been tempted to claim that we strictly speaking see only the surfaces of opaque objects, a diagnosis that \citet{Martin:2008kl} argues is applicable to Noë's case as well. Noë never explicitly articulates this assumption (though he comes close at places). Moreover, \citet[698-699]{Noe:2008oq} explicitly denies the assumption in his response to Martin. However, as Quine continually reminds us, belief is one thing, and commitment quite another. There is simply no way to generate the alleged prima facie conflict in perceptual content without making some such assumption.

Consider again the way the color of the specular highlight obscures the color of the corresponding part of the surface. I see the color of the the specular highlight. I see the unform color of the surface. So far there is no conflict. In order to generate the prima facie conflict in perceptual content, one would need to assume, in addition, that one can only see the color of a surface by seeing the color of all its parts. But if that assumption, along with the evident facts of perception (that I see the color of the specular highlight and I see the color of the surface), is what generates the prima facie conflict in perceptual content, then we should reject that assumption. Moreover, rejecting that assumption does not require that we attribute to color experience a dual content.

% subsection specular_highlights (end)

\subsection{Uneven Illumination} % (fold)
\label{sub:uneven_illumination}

Shadows and specular highlights are both what \citet{Johnston:1992ck} calls transient colors. However, color constancy phenomena need not involve the contrast between steady and transient colors. 

The color of a surface can appear different in different scenes and conditions of illumination because of the differently distributed highlights, reflections, and shadows, but not every difference in appearance can be explained as a difference in transient color. Suppose that we are looking at a red chip with a matte surface, unshadowed, in diffuse light, in a monochromatic environment. If we dim the light somewhat but within the bounds of ``normality'', then the character of our experience will vary. This is a case of color constancy---the color of the chip appears unaltered through the course of phenomenally distinct experiences---but the phenomenal difference is not due to a difference in transient color for none are present  \citep[see][]{Kalderon:2006fk}.

That was an example of successive color constancy not involving transient color. There are examples of simultaneous color constancy not involving transient color as well.

% subsection uneven_illumination (end)

% section stability_and_flux (end)


\section{Illumination Insensitivity and the Vestigal Remnants of Inner Representation} % (fold)
\label{sec:vestigal_remnants_of_inner_representations}

One of the many salutary aspects of \emph{Action in Perception} is Noë's trenchant opposition to the inner representation model of experience. However, philosophical pictures have a life of their own and are not so easily quelled. Indeed, many of the problems of the previous section are due to the lingering effects of the inner representation model.

Experience, as conceived by Noë, is illumination insensitive in the following sense: Noë is implicitly committed to our not strictly speaking seeing illumination properties (which is not, of course, to say that we do not figure in our experience). Moreover, this illumination insensitivity is arguably a vestigal remnant of inner representation. For on Noë's account, experience is, if not an inner picture, then, in an important respect, picture-like---illumiation properties figure in our experience the way that they are depicted by pictures. % We see the colors on the surface of a picture and only therbey come to experience the color the illuminant is depicted as having. Similarly, on Noë's account, we see the aaprent colors of surfaces and only thereby come to see the color of the illuminant.



% section vestigal_remnants_of_inner_representations (end)

\section{The Problem of Unity} % (fold)
\label{sec:the_problem_of_unity}



% section the_problem_of_unity (end)

\section{Naïve Realism and the Partiality of Perception} % (fold)
\label{sec:naïve_realism_and_the_partiality_of_perception}



% section naïve_realism_and_the_partiality_of_perception (end)

% Bibligography
\bibliographystyle{plainnat} 
\bibliography{Philosophy.bib} 

\end{document}
